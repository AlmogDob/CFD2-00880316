\documentclass[11pt, a4paper]{article}

\usepackage{amsmath, amssymb, titling}
\usepackage[margin=3cm]{geometry}
\usepackage[colorlinks=true, linkcolor=black, urlcolor=black, citecolor=black]{hyperref}
\usepackage{graphicx}
\usepackage{float}
\usepackage{fancyhdr, lastpage}
\usepackage{xcolor}

\renewcommand\maketitlehooka{\null\mbox{}\vfill}
\renewcommand\maketitlehookd{\vfill\null}

\title{Computational Fluid Dynamics \\ HW1}
\author{Almog Dobrescu\\\\ID 214254252}

\pagestyle{fancy}
\cfoot{Page \thepage\ of \pageref{LastPage}}

\begin{document}

\maketitle
\thispagestyle{empty}
\newpage

\setcounter{page}{1}
\tableofcontents
\vfil
\listoffigures
\newpage

\section{Inviscid Burgers Equation}
The Inviscid Burgers equation, in conservation law form, is given by:
\begin{equation}
    \begin{array}{cc}
        \displaystyle\frac{\partial u}{\partial t} + \frac{\partial F}{\partial x} = 0 & F = \displaystyle F_{(u)} = \frac{u^2}{2}
    \end{array}
\end{equation} 
In non-conservation law form, is given by:
\begin{equation}
    \begin{array}{cc}
        \displaystyle\frac{\partial u}{\partial t} + A\frac{\partial u}{\partial x} = 0 & A = \displaystyle \frac{\partial F}{\partial u} = u
    \end{array}
\end{equation}
The equation is obtained by neglecting the viscous term from the viscous Burger equation.

\subsection{Boundary and Initial Conditions}
\begin{equation}
    \begin{array}{lcl}
        u_{(x=0,t)} & = & 1.0 \\
        u_{(x=1,t)} & = & u_1 \\
        u_{(x,t=0)} & = & 1-(1-u_1)\cdot x
    \end{array}
\end{equation}

\subsection{Finite Volume Formulation}
\begin{equation}
    u_i^{n+1}=u_i^n-\frac{\Delta t}{\Delta x}\left(f_{i+\frac{1}{2}}^n-f_{i-\frac{1}{2}}^n\right)
\end{equation}
\begin{itemize}
    \item For first-order schemes, there is no variation within a cell, and the value there is constant.
    \item For second-order schemes, the variation within the cell is linear.  \end{itemize}


\subsection{First Order Roe Method $(u_1 = 0.0)$}
Roe scheme is based on the solution of the linear problem:
\begin{equation}
        \displaystyle\frac{\partial u}{\partial t} + \bar{A}\frac{\partial u}{\partial x} = 0 
\end{equation}
Where $\bar{A}$ is a constant matrix that is dependent on local conditions. The matrix is constructed in a way to guarantee unifrom validity across discontinuities: 
\begin{enumerate}
    \item For any $u_i$, $u_{i+1}$:\begin{equation*}
        F_{i+1}-F_{i} = \bar{A}\cdot\left(u_{i+1}-u_i\right)
    \end{equation*}
    \item When $u=u_i=u_{i+1}$ then:\begin{equation*}
        \bar{A}_{\left(u_i,u_{i+1}\right)}=\bar{A}_{\left(u,u\right)}=\frac{\partial F}{\partial u}=u
    \end{equation*}
\end{enumerate}
In case of the Burgers equation, the matrix $\bar{A}$ is a scalar, namely, $\bar{A}=\bar{u}$. The equation becomes:
\begin{equation}
        \displaystyle\frac{\partial u}{\partial t} + \bar{u}\frac{\partial u}{\partial x} = 0 
\end{equation}
The value of $\bar{u}$ for the cell face between \emph{i} and \emph{i+1} is determined from the first conditions:
\begin{equation}
    \bar{u}=\bar{u}_{i+\frac{1}{2}}=\frac{F_{i+1}-F_i}{u_{i+1}-u_i}=\frac{\displaystyle\frac{1}{2}u_{i+1}^2-\frac{1}{2}u_i^2}{u_{i+1}-u_i}=\left\{\begin{array}{cc}
        \displaystyle\frac{u_i+u_{i+1}}{2} & u_i\neq u_{i+1} \\
        u_i & u_i=u_{i+1} 
    \end{array}\right.
\end{equation}
The single wave that emanates from the cell interface travels either in the positive or negative direction, depending upon the sighn of $\bar{u}_{i+\frac{1}{2}}$. Define:
\begin{equation}
    \left\{\begin{array}{cc}
        \begin{array}{c}
            \bar{u}_{i+\frac{1}{2}}^+\triangleq\displaystyle\frac{1}{2}\left(\bar{u}_{i+\frac{1}{2}}+\left|\bar{u}_{i+\frac{1}{2}}\right|\right)\geq0 \\\\
            \bar{u}_{i+\frac{1}{2}}^-\triangleq\displaystyle\frac{1}{2}\left(\bar{u}_{i+\frac{1}{2}}-\left|\bar{u}_{i+\frac{1}{2}}\right|\right)\leq0 \\
        \end{array} & \bar{u}_{i+\frac{1}{2}}=\bar{u}_{i+\frac{1}{2}}^++\bar{u}_{i+\frac{1}{2}}^-
    \end{array}\right.
\end{equation}
Using the jump relation, the numerical flux at the cell interface can be evaluated by one of the following:
\begin{equation}
    \left\{\begin{array}{l}
        f_{i+\frac{1}{2}}-F_i = \bar{u}_{i+\frac{1}{2}}^-\cdot\left(u_{i+1}-u_i\right) \\\\
        F_{i+1}-f_{i+\frac{1}{2}}=\bar{u}_{i+\frac{1}{2}}^+\cdot\left(u_{i+1}-u_i\right)
    \end{array}\right.
\end{equation}
The numerical flux may then be written in the following symmetric from:
\begin{equation}
    \begin{array}{l}
        \displaystyle f_{i+\frac{1}{2}}=\frac{F_i+F_{i+1}}{2}-\frac{1}{2}\left(\bar{u}_{i+\frac{1}{2}}^+-\bar{u}_{i+\frac{1}{2}}^-\right)\left(u_{i+1}-u_i\right) \\
        \mathrm{OR:} \\
        \displaystyle f_{i+\frac{1}{2}}=\frac{F_i+F_{i+1}}{2}-\frac{1}{2}\left|\bar{u}_{i+\frac{1}{2}}\right|\left(u_{i+1}-u_i\right)
    \end{array}
\end{equation}
Since Roe's scheme can't distinguish between the types of discontinuity, it may result in an expansion shock where the analytical solution is an expansion wave. To guarantee a physical solution the scheme will be modified like so:\\
Define 
\begin{equation*}
    \varepsilon=\max\left(0,\frac{u_{i+1}-u_i}{2}\right)
\end{equation*}
The interface wave speed becomes
\begin{equation}
    \bar{u}_{i+\frac{1}{2}}=\left\{\begin{array}{ccc}
        \bar{u}_{i+\frac{1}{2}} & \bar{u}_{i+\frac{1}{2}}\geq\varepsilon & \mathrm{compression} \\
        \varepsilon & \bar{u}_{i+\frac{1}{2}}<\varepsilon & \mathrm{expansion}
    \end{array}\right.
\end{equation}


\subsection{Second Order Roe $(u_1 = 0.5)$}
The first-order accurate Roe method interface flux function will be denoted like this:
\begin{equation*}
    f_{i+\frac{1}{2}}^{\mathrm{Roe},1}=f_{\left(u_i, u_{i+1}\right)}
\end{equation*}
The second order accurate Roe takes the form:
\begin{equation*}
    f_{i+\frac{1}{2}}^{\mathrm{Roe},2}=f_{\left(u_{i+1}^l, u_{i+1}^r\right)}
\end{equation*}
Hence:
\begin{equation}
    \begin{matrix}
        \displaystyle f_{i+\frac{1}{2}}^{\mathrm{Roe},2}=\frac{1}{2}\left(F_{\left(u_{1+\frac{1}{2}}^l\right)}+F_{\left(u_{1+\frac{1}{2}}^r\right)}-\left|\bar{u}_{i+\frac{1}{2}}\right|\left(u_{1+\frac{1}{2}}^r-u_{1+\frac{1}{2}}^l\right)\right) \\\\
        \displaystyle \bar{u}_{1+\frac{1}{2}}=\frac{F_{\left(u_{1+\frac{1}{2}}^r\right)}-F_{\left(u_{1+\frac{1}{2}}^l\right)}}{u_{1+\frac{1}{2}}^r-u_{1+\frac{1}{2}}^l}
    \end{matrix}
\end{equation}

\subsubsection{Without Limiters}
The interface values without limiters are evaluated as:
\begin{equation}
    \begin{matrix}
        \left\{\begin{array}{lcl}
            u_{i+\frac{1}{2}}^l & = & u_i+\frac{1-k}{4}\delta u_{i-\frac{1}{2}}+\frac{1+k}{4}\delta u_{i+\frac{1}{2}} \\
            u_{i+\frac{1}{2}}^r & = & u_{i+1}-\frac{1+k}{4}\delta u_{i+\frac{1}{2}}-\frac{1-k}{4}\delta u_{i+\frac{3}{2}}
        \end{array}\right. && \delta u_i\triangleq u_{i+\frac{1}{2}} - u_{i-\frac{1}{2}}
    \end{matrix}
\end{equation}
The parameter \emph{k} determines the scheme:
\begin{equation*}
    k = \left\{\begin{array}{cr}
        -1 & \mathrm{upwind} \\
        1 & \mathrm{central}
    \end{array}\right.
\end{equation*}

\subsubsection{With Limiters}
The interface values with limiters are evaluated as:
\begin{equation}
    \begin{matrix}
        \left\{\begin{array}{lcl}
            u_{i+\frac{1}{2}}^l & = & u_i+\frac{1-k}{4}\overline{\delta^+} u_{i-\frac{1}{2}}+\frac{1+k}{4}\overline{\delta^-} u_{i+\frac{1}{2}} \\
            u_{i+\frac{1}{2}}^r & = & u_{i+1}-\frac{1+k}{4}\overline{\delta^+} u_{i+\frac{1}{2}}-\frac{1-k}{4}\overline{\delta^-} u_{i+\frac{3}{2}}
        \end{array}\right. && {\overline{\delta^\pm}} u\text{ are limited slopes}
    \end{matrix}
\end{equation}

\newpage
\section{Generalized Burgers Equation}



\end{document}