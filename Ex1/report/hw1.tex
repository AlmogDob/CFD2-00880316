\documentclass[11pt, a4paper]{article}

\usepackage{amsmath, amssymb, titling}
\usepackage[margin=3cm]{geometry}
\usepackage[colorlinks=true, linkcolor=black, urlcolor=black, citecolor=black]{hyperref}
\usepackage{graphicx}
\usepackage{float}
\usepackage{fancyhdr, lastpage}
\usepackage{xcolor}

\renewcommand\maketitlehooka{\null\mbox{}\vfill}
\renewcommand\maketitlehookd{\vfill\null}

\title{Computational Fluid Dynamics \\ HW1}
\author{Almog Dobrescu\\\\ID 214254252}

\pagestyle{fancy}
\cfoot{Page \thepage\ of \pageref{LastPage}}

\begin{document}

\maketitle
\thispagestyle{empty}
\newpage

\setcounter{page}{1}
\tableofcontents
\vfil
\listoffigures
\newpage

\section{Inviscid Burgers Equation}
The Inviscid Burgers equation, in conservation law form, is given by:
\begin{equation}
    \begin{array}{cc}
        \displaystyle\frac{\partial u}{\partial t} + \frac{\partial F}{\partial x} = 0 & F = \displaystyle F_{(u)} = \frac{u^2}{2}
    \end{array}
\end{equation} 
In non-conservation law form, is given by:
\begin{equation}
    \begin{array}{cc}
        \displaystyle\frac{\partial u}{\partial t} + A\frac{\partial u}{\partial x} = 0 & A = \displaystyle \frac{\partial F}{\partial u} = u
    \end{array}
\end{equation}
The equation is obtained by neglecting the viscous term from the viscous Burger equation.

\subsection{Boundary and Initial Conditions}
\begin{equation}
    \begin{array}{lcl}
        u_{(x=0,t)} & = & 1.0 \\
        u_{(x=1,t)} & = & u_1 \\
        u_{(x,t=0)} & = & 1-(1-u_1)\cdot x
    \end{array}
\end{equation}

\subsection{Finite Volume Formulation}
\begin{equation}
    u_i^{n+1}=u_i^n-\frac{\Delta t}{\Delta x}\left(f_{i+\frac{1}{2}}^n-f_{i-\frac{1}{2}}^n\right)
\end{equation}
\begin{itemize}
    \item For first-order schemes, there is no variation within a cell, and the value there is constant.
    \item For second-order schemes, the variation within the cell is linear. 
\end{itemize}

\subsection{CFL number}
For the Roe method, the CFL number is defined as:
\begin{equation}
    \mathrm{CFL}=\frac{u\Delta t}{\Delta x}
\end{equation}
We will want to set the maximal value of the \emph{CFL} number. We will find the $\Delta t$ at each cell and $\left(\Delta t_i\right)$ and set the $\Delta t$ of the current step as:
\begin{equation}
    \Delta t=\min\left(\Delta t_i\right)\ \forall i
\end{equation}

\subsection{First Order Roe Method $(u_1 = 0.0)$}
\label{Roe_first}
Roe scheme is based on the solution of the linear problem:
\begin{equation}
        \displaystyle\frac{\partial u}{\partial t} + \bar{A}\frac{\partial u}{\partial x} = 0 
\end{equation}
Where $\bar{A}$ is a constant matrix that is dependent on local conditions. The matrix is constructed in a way that guarantees uniform validity across discontinuities: 
\begin{enumerate}
    \item For any $u_i$, $u_{i+1}$:\begin{equation*}
        F_{i+1}-F_{i} = \bar{A}\cdot\left(u_{i+1}-u_i\right)
    \end{equation*}
    \item When $u=u_i=u_{i+1}$ then:\begin{equation*}
        \bar{A}_{\left(u_i,u{i+1}\right)}=\bar{A}_{\left(u,u\right)}=\frac{\partial F}{\partial u}=u
    \end{equation*}
\end{enumerate}
In the case of the Burgers equation, the matrix $\bar{A}$ is a scalar, namely, $\bar{A}=\bar{u}$. The equation becomes:
\begin{equation}
        \displaystyle\frac{\partial u}{\partial t} + \bar{u}\frac{\partial u}{\partial x} = 0 
\end{equation}
The value of $\bar{u}$ for the cell face between \emph{i} and \emph{i+1} is determined from the first conditions:
\begin{equation}
    \bar{u}=\bar{u}_{i+\frac{1}{2}}=\frac{F_{i+1}-F_i}{u_{i+1}-u_i}=\frac{\displaystyle\frac{1}{2}u_{i+1}^2-\frac{1}{2}u_i^2}{u_{i+1}-u_i}=\left\{\begin{array}{cc}
        \displaystyle\frac{u_i+u_{i+1}}{2} & u_i\neq u_{i+1} \\
        u_i & u_i=u_{i+1} 
    \end{array}\right.
\end{equation}
The single wave that emanates from the cell interface travels either in the positive or negative direction, depending upon the sign of $\bar{u}_{i+\frac{1}{2}}$. Define:
\begin{equation}
    \left\{\begin{array}{cc}
        \begin{array}{c}
            \bar{u}_{i+\frac{1}{2}}^+\triangleq\displaystyle\frac{1}{2}\left(\bar{u}_{i+\frac{1}{2}}+\left|\bar{u}_{i+\frac{1}{2}}\right|\right)\geq0 \\\\
            \bar{u}_{i+\frac{1}{2}}^-\triangleq\displaystyle\frac{1}{2}\left(\bar{u}_{i+\frac{1}{2}}-\left|\bar{u}_{i+\frac{1}{2}}\right|\right)\leq0 \\
        \end{array} & \bar{u}_{i+\frac{1}{2}}=\bar{u}_{i+\frac{1}{2}}^++\bar{u}_{i+\frac{1}{2}}^-
    \end{array}\right.
\end{equation}
Using the jump relation, the numerical flux at the cell interface can be evaluated by one of the following:
\begin{equation}
    \left\{\begin{array}{l}
        f_{i+\frac{1}{2}}-F_i = \bar{u}_{i+\frac{1}{2}}^-\cdot\left(u_{i+1}-u_i\right) \\\\
        F_{i+1}-f_{i+\frac{1}{2}}=\bar{u}_{i+\frac{1}{2}}^+\cdot\left(u_{i+1}-u_i\right)
    \end{array}\right.
\end{equation}
The numerical flux may then be written in the following symmetric form:
\begin{equation}
    \begin{array}{l}
        \displaystyle f_{i+\frac{1}{2}}=\frac{F_i+F_{i+1}}{2}-\frac{1}{2}\left(\bar{u}_{i+\frac{1}{2}}^+-\bar{u}_{i+\frac{1}{2}}^-\right)\left(u_{i+1}-u_i\right) \\
        \mathrm{OR:} \\
        \displaystyle f_{i+\frac{1}{2}}=\frac{F_i+F_{i+1}}{2}-\frac{1}{2}\left|\bar{u}_{i+\frac{1}{2}}\right|\left(u_{i+1}-u_i\right)
    \end{array}
\end{equation}

% Since Roe's scheme can't distinguish between the types of discontinuity, it may result in an expansion shock where the analytical solution is an expansion wave. To guarantee a physical solution the scheme will be modified like so:\\
% Define 
% \begin{equation*}
%     \varepsilon=\max\left(0,\frac{u_{i+1}-u_i}{2}\right)
% \end{equation*}
% The interface wave speed becomes
% \begin{equation}
%     \bar{u}_{i+\frac{1}{2}}=\left\{\begin{array}{ccc}
%         \bar{u}{i+\frac{1}{2}} & \bar{u}{i+\frac{1}{2}}\geq\varepsilon & \mathrm{compression} \\
%         \varepsilon & \bar{u}_{i+\frac{1}{2}}<\varepsilon & \mathrm{expansion}
%     \end{array}\right.
% \end{equation}

\subsubsection{Effect of CFL}


\subsection{Second Order Roe $(u_1 = 0.5)$}
The first-order accurate Roe method interface flux function will be denoted like this:
\begin{equation*}
    f_{i+\frac{1}{2}}^{\mathrm{Roe},1}=f_{\left(u_i, u_{i+1}\right)}
\end{equation*}
The second order accurate Roe takes the form:
\begin{equation*}
    f_{i+\frac{1}{2}}^{\mathrm{Roe},2}=f_{\left(u_{i+1}^l, u_{i+1}^r\right)}
\end{equation*}
Hence:
\begin{equation}
    \begin{matrix}
        \displaystyle f_{i+\frac{1}{2}}^{\mathrm{Roe},2}=\frac{1}{2}\left(F_{\left(u_{1+\frac{1}{2}}^l\right)}+F_{\left(u_{1+\frac{1}{2}}^r\right)}-\left|\bar{u}_{i+\frac{1}{2}}\right|\left(u_{1+\frac{1}{2}}^r-u_{1+\frac{1}{2}}^l\right)\right) \\\\
        \displaystyle \bar{u}_{1+\frac{1}{2}}=\frac{F_{\left(u_{1+\frac{1}{2}}^r\right)}-F_{\left(u_{1+\frac{1}{2}}^l\right)}}{u_{1+\frac{1}{2}}^r-u_{1+\frac{1}{2}}^l} = \frac{u_{i+\frac{1}{2}}^l+u_{i+\frac{1}{2}}^r}{2}
    \end{matrix}
\end{equation}

\subsubsection{Without Limiters}
The interface values without limiters are evaluated as:
\begin{equation}
    \begin{matrix}
        \left\{\begin{array}{lcl}
            u_{i+\frac{1}{2}}^l & = & \displaystyle u_i+\frac{1-k}{4}\delta u_{i-\frac{1}{2}}+\frac{1+k}{4}\delta u_{i+\frac{1}{2}} \\\\
            u_{i+\frac{1}{2}}^r & = & \displaystyle u_{i+1}-\frac{1+k}{4}\delta u_{i+\frac{1}{2}}-\frac{1-k}{4}\delta u_{i+\frac{3}{2}}
        \end{array}\right. && \delta u_i\triangleq u_{i+\frac{1}{2}} - u_{i-\frac{1}{2}}
    \end{matrix}
\end{equation}
The parameter \emph{k} determines the scheme:
\begin{equation*}
    k = \left\{\begin{array}{cr}
        -1 & \mathrm{upwind} \\
        1 & \mathrm{central}
    \end{array}\right.
\end{equation*}

\subsubsection{With Limiters}
The interface values with limiters are evaluated as:
\begin{equation}
    \begin{matrix}
        \left\{\begin{array}{lcl}
            u_{i+\frac{1}{2}}^l & = & u_i+\frac{1-k}{4}\overline{\delta^+} u_{i-\frac{1}{2}}+\frac{1+k}{4}\overline{\delta^-} u_{i+\frac{1}{2}} \\
            u_{i+\frac{1}{2}}^r & = & u_{i+1}-\frac{1+k}{4}\overline{\delta^+} u_{i+\frac{1}{2}}-\frac{1-k}{4}\overline{\delta^-} u_{i+\frac{3}{2}}
        \end{array}\right. && {\overline{\delta^\pm}} u\text{ are limited slopes}
    \end{matrix}
\end{equation}
$\overline{\delta}$ is an operator such that $\overline{\delta}u_i=\psi\delta u_i$, where $\psi\left(r\right)$ is a limiter function and:
\begin{equation}
    r^\pm=\left\{\begin{array}{ccc}
        r_{1+\frac{1}{2}}^+ & \triangleq & \displaystyle\frac{u_{i+2}-u_{i+1}}{u_{i+1}-u_i}=\frac{\Delta u_{i+1}}{\Delta u_i} \\
        r_{1+\frac{1}{2}}^- & \triangleq & \displaystyle\frac{u_{i}-u_{i-1}}{u_{i+1}-u_i}=\frac{\nabla u_{i}}{\nabla u_{i+1}}
    \end{array}\right.
\end{equation}
There are many types of limiters. For example, the van Albada limiter:
\begin{equation}
    \psi\left(r\right)=\frac{r+r^2}{1+r^2}
\end{equation}

\subsubsection{Effect of CFL}

\subsubsection{Effect of Limiter}

\newpage
\section{Generalized Burgers Equation}
The generalized Burgers equation is given by:
\begin{equation}
    \frac{\partial u}{\partial t}+\left(c+bu\right)\frac{\partial u}{\partial x}=\mu\frac{\partial^2u}{\partial x^2}
\end{equation}
Where:
\begin{equation*}
    \begin{matrix}
        c=\frac{1}{2} && b=-1 && \mu=\left[0.001, 0.25\right]
    \end{matrix}
\end{equation*}
The equation can also be presented as:
\begin{equation}
    \begin{matrix}
        \displaystyle\frac{\partial u}{\partial t}+\frac{\partial\bar{F}}{\partial x}=0 && \bar{F}=\underbrace{cu+\frac{bu^2}{2}}_F-\underbrace{\mu\frac{\partial u}{\partial x}}_{F_\nu}
    \end{matrix}
\end{equation}
In non-conservation law form, is given by:
\begin{equation}
    \begin{matrix}
        \displaystyle\frac{\partial u}{\partial t}+A\frac{\partial u}{\partial x}=0 && \displaystyle A=\frac{\partial\bar{F}}{\partial u}=c+bu-\mu\frac{\partial}{\partial u}\left(\frac{\partial u}{\partial x}\right)
    \end{matrix}
\end{equation}
The generalized Burgers equation has a stationary solution:
\begin{equation}
    u=-\frac{c}{b}\left(1+\tanh{\left(\displaystyle\frac{c\left(x-x_0\right)}{2\mu}\right)}\right)
\end{equation}

\subsection{Domain and Computational Mesh}
Using 41 grid points with $\Delta x=1$ and computing until $t=18.0$. $\Delta t=\left[0.5, 1.0\right]$.

\subsection{Boundary and Initial Conditions}
\subsubsection{Initial Conditions}
\begin{equation}
    u_{\left(x,t=0\right)}=\frac{1}{2}\left(1+\tanh\left(250\left(x-20\right)\right)\right)
\end{equation}
\subsubsection{Boundary Conditions}
Using Dirichlet boundary conditions:
\begin{equation}
    \begin{matrix}
        \displaystyle u_{\left(x=0,t\right)}=0 && \displaystyle u_{\left(x=40,t\right)}=1
    \end{matrix}
\end{equation}

\subsection{First Order Roe Method (explicit)}
As written above for the inviscid Burgers equation (\ref{Roe_first}), Roes scheme is based on the solution of the linear problem:
\begin{equation}
    \begin{matrix}
        \displaystyle\frac{\partial u}{\partial t} + \bar{A}\frac{\partial u}{\partial x}=\mu\frac{\partial^2u}{\partial x^2} && \displaystyle \bar{A}=\frac{\partial F}{\partial u}
    \end{matrix}         
\end{equation}
In the case of the Burgers equation, the matrix $\bar{A}$ is a scalar.
\begin{equation}
    \bar{A}=\bar{A}_{i+\frac{1}{2}}=\frac{F_{i+1}-F_i}{u_{i+1}-u_i}=\left\{\begin{array}{cc} \displaystyle\frac{\displaystyle c\left(u_{i+1}-u_i\right)+\frac{b}{2}\left(u_{i+1}^2-u_i^2\right)}{u_{i+1}-u_i} & u_i\neq u_{i+1} \\ 
    A_i & u_i=u_{i+1} \end{array}\right.  
\end{equation} 
The numerical flux at the cell interface:
\begin{equation}
    \displaystyle \bar{f}_{i+\frac{1}{2}}=\frac{F_i+F_{i+1}}{2}-\frac{1}{2}\left(\bar{A}_{i+\frac{1}{2}}^+-\bar{A}_{i+\frac{1}{2}}^-\right)\left(u_{i+1}-u_i\right)
\end{equation}
Where:
\begin{equation}
    \left\{\begin{array}{cc}
        \begin{array}{c}
            \bar{A}_{i+\frac{1}{2}}^+\triangleq\displaystyle\frac{1}{2}\left(\bar{A}{i+\frac{1}{2}}+\left|\bar{A}_{i+\frac{1}{2}}\right|\right)\geq0 \\\\
            \bar{A}_{i+\frac{1}{2}}^-\triangleq\displaystyle\frac{1}{2}\left(\bar{A}_{i+\frac{1}{2}}-\left|\bar{A}_{i+\frac{1}{2}}\right|\right)\leq0 \\
        \end{array} & \bar{A}_{i+\frac{1}{2}}=\bar{A}_{i+\frac{1}{2}}^++\bar{A}_{i+\frac{1}{2}}^-
    \end{array}\right.
\end{equation}

% In the case of the Burgers equation, the matrix $\bar{A}$ is a scalar, namely, $\bar{A}=\bar{u}$. The equation becomes: \begin{equation} \displaystyle\frac{\partial u}{\partial t} + \bar{u}\frac{\partial u}{\partial x}=\mu\frac{\partial^2u}{\partial x^2} \end{equation} \begin{equation} \bar{u}=\bar{u}_{i+\frac{1}{2}}=\frac{F_{i+1}-F_i}{u_{i+1}-u_i}=\left\{\begin{array}{cc} \displaystyle\frac{\displaystyle c\left(u_{i+1}-u_i\right)+\frac{b}{2}\left(u_{i+1}^2-u_i^2\right)}{u_{i+1}-u_i} & u_i\neq u_{i+1} \\ u_i & u_i=u_{i+1} \end{array}\right.  \end{equation} The numerical flux at the cell interface:
% \begin{equation}
%     \displaystyle \bar{f}_{i+\frac{1}{2}}=\frac{F_i+F_{i+1}}{2}-\frac{1}{2}\left(\bar{u}_{i+\frac{1}{2}}^+-\bar{u}_{i+\frac{1}{2}}^-\right)\left(u_{i+1}-u_i\right)
% \end{equation}
% Where:
% \begin{equation}
%     \left\{\begin{array}{cc}
%         \begin{array}{c}
%             \bar{u}_{i+\frac{1}{2}}^+\triangleq\displaystyle\frac{1}{2}\left(\bar{u}{i+\frac{1}{2}}+\left|\bar{u}_{i+\frac{1}{2}}\right|\right)\geq0 \\\\
%             \bar{u}_{i+\frac{1}{2}}^-\triangleq\displaystyle\frac{1}{2}\left(\bar{u}_{i+\frac{1}{2}}-\left|\bar{u}_{i+\frac{1}{2}}\right|\right)\leq0 \\
%         \end{array} & \bar{u}_{i+\frac{1}{2}}=\bar{u}_{i+\frac{1}{2}}^++\bar{u}_{i+\frac{1}{2}}^-
%     \end{array}\right.
% \end{equation}
And finally:
\begin{equation}
    u_i^{n+1}=u_i^n-\frac{\Delta t}{\Delta x}\left(\bar{f}_{i+\frac{1}{2}}^n-\bar{f}_{i-\frac{1}{2}}^n\right)+\mu\frac{\Delta t}{\left(\Delta x\right)^2}\left(u_{i+1}^n-2u_i^n+u_{i-1}^n\right)
\end{equation}


% MY WAY

% As written above for the inviscid Burgers equation (\ref{Roe_first}), Roes scheme is based on the solution of the linear problem:
% \begin{equation}
%         \displaystyle\frac{\partial u}{\partial t} + \bar{A}\frac{\partial u}{\partial x} = 0 
% \end{equation}
% Where $\bar{A}$ is a constant matrix that is dependent on local conditions. The matrix is constructed in a way that guarantees uniform validity across discontinuities: 
% \begin{enumerate}
%     \item For any $u_i$, $u_{i+1}$:\begin{equation*}
%         \bar{F}{i+1}-\bar{F}{i} = \bar{A}\cdot\left(u_{i+1}-u_i\right)
%     \end{equation*}
%     \item When $u=u_i=u_{i+1}$ then:\begin{equation*}
%         \bar{A}{\left(u_i,u{i+1}\right)}=\bar{A}_{\left(u,u\right)}=\frac{\partial \bar{F}}{\partial u}
%     \end{equation*}
% \end{enumerate}
% In the case of the Burgers equation, the matrix $\bar{A}$ is a scalar. The value of $\bar{A}$ for the cell face between \emph{i} and \emph{i+1} is determined from the first conditions, using central difference for the viscus term (forward/backward at the edges):
% \begin{equation}
%     \begin{array}{c}
%         \bar{A}=\bar{A}{i+\frac{1}{2}}=\displaystyle\frac{\bar{F}{i+1}-\bar{F}i}{u{i+1}-u_i}=\frac{\displaystyle c\left(u_{i+1}-u_i\right)+\frac{b}{2}\left(u_{i+1}^2-u_{i}^2\right)-\mu\left(\left.\frac{\partial u}{\partial x}\right|{i+1}-\left.\frac{\partial u}{\partial x}\right|{i}\right)}{u_{i+1}-u_i}
%         % \\ \Downarrow \\
%         % A=\frac{\displaystyle c\left(u_{i+1}-u_i\right)+\frac{b}{2}\left(u_{i+1}^2-u_{i}^2\right)-\mu\left(\left.\frac{\partial u}{\partial x}\right|{i+1}-\left.\frac{\partial u}{\partial x}\right|{i}\right)}{u_{i+1}-u_i}
%     \end{array}   
% \end{equation}
% The numerical flux at the cell interface:
% \begin{equation}
%     \displaystyle \bar{f}{i+\frac{1}{2}}=\frac{\bar{F}_i+\bar{F}{i+1}}{2}-\frac{1}{2}\left(\bar{A}{i+\frac{1}{2}}^+-\bar{A}{i+\frac{1}{2}}^-\right)\left(u_{i+1}-u_i\right)
% \end{equation}
% Where:
% \begin{equation}
%     \left\{\begin{array}{cc}
%         \begin{array}{c}
%             \bar{A}{i+\frac{1}{2}}^+\triangleq\displaystyle\frac{1}{2}\left(\bar{A}{i+\frac{1}{2}}+\left|\bar{A}_{i+\frac{1}{2}}\right|\right)\geq0 \\\\
%             \bar{A}{i+\frac{1}{2}}^-\triangleq\displaystyle\frac{1}{2}\left(\bar{A}{i+\frac{1}{2}}-\left|\bar{A}_{i+\frac{1}{2}}\right|\right)\leq0 \\
%         \end{array} & \bar{A}{i+\frac{1}{2}}=\bar{A}{i+\frac{1}{2}}^++\bar{A}_{i+\frac{1}{2}}^-
%     \end{array}\right.
% \end{equation}
% And finally:
% \begin{equation}
%     u_i^{n+1}=u_i^n-\frac{\Delta t}{\Delta x}\left(\bar{f}{i+\frac{1}{2}}^n-\bar{f}{i-\frac{1}{2}}^n\right)
% \end{equation}

\subsection{MacCormack Method}
The original MacCormack method applied to Burgers equaiton results in:
\begin{equation}
    \begin{array}{cc}
        \displaystyle\mathrm{Predictor:} & \displaystyle u_i^{\overline{n+1}}=u_i^n-\Delta t\frac{\Delta F_i^n}{\Delta x}+r\delta^2u_i^n \\\\ 
        \displaystyle\mathrm{Corrector:} & \displaystyle u_i^{n+1}=\frac{1}{2}\left(u_i^n+u_i^{\overline{n+1}}-\Delta t\frac{\nabla F_i^{\overline{n+1}}}{\Delta x}\right)+r\delta^2u_i^{\overline{n+1}}
    \end{array}
\end{equation}
Where:
\begin{itemize}
    \item $\displaystyle r=\frac{\mu\Delta t}{\left(\Delta x\right)^2}$
    \item $\displaystyle \delta^2u_i=u_{i+1}-2u_i+u_{i-1}$
    \item $\Delta f=f_{i+1}-f_i$
    \item $\nabla f=f_i-f_{i-1}$
\end{itemize}
% \begin{equation*}
%     \Downarrow
% \end{equation*}
% \begin{equation}
%     \begin{array}{cc}
%         \displaystyle\mathrm{Predictor:} & \displaystyle u_i^{\overline{n+1}}=u_i^n-\Delta t\frac{F_{i+1}^n-F_{i}^n}{\Delta x}+r\delta^2u_i^n \\\\ 
%         \displaystyle\mathrm{Corrector:} & \displaystyle u_i^{n+1}=\frac{1}{2}\left(u_i^n+u_i^{\overline{n+1}}-\Delta t\frac{\nabla F_i^{\overline{n+1}}}{\Delta x}\right)+r\delta^2u_i^{\overline{n+1}}
%     \end{array}
% \end{equation}

\subsubsection{Stability}
The stability condition for MacCormack's method is:
\begin{equation}
    \Delta t\le\frac{\left(\Delta x\right)^2}{\left|u\right|\Delta x+\mu}
\end{equation}

\subsection{Beam and Warming}
Beam and Warming inroduced the Delta form and their method is more efficient:
\begin{equation}
    \begin{array}{c}
        \displaystyle\left(I+\theta\left(\Delta t\frac{D_0A_i^n}{2\Delta x}-\frac{\Delta t\mu}{\left(\Delta x\right)^2}\delta^2\right)\right)\Delta u_i^n=\underbrace{-\Delta t\frac{F_{i+1}^n-F_{i-1}^n}{2\Delta x}+\Delta t\mu\frac{u^n_{i+1}-2u^n_i+u^n_{i-1}}{\left(\Delta x\right)^2}}_{\displaystyle\mathrm{RHS}_i^n} \\
        \left\{\begin{array}{ll}
            \theta=1 & \mathrm{first\ order} \\
            \theta=0.5 & \mathrm{second\ order}
        \end{array}\right.
    \end{array}
\end{equation}
Deriving the matrix to invert:
\begin{align*}
    \displaystyle\left(I+\theta\Delta t\frac{D_0A_i^n}{2\Delta x}\right)\Delta u_i^n-\theta\frac{\Delta t\mu}{\left(\Delta x\right)^2}\delta^2\Delta u^n_i&=\mathrm{RHS}^n_i\\
    \displaystyle\left(I+\theta\Delta t\frac{D_0A_i^n}{2\Delta x}\right)\Delta u_i^n-\theta\frac{\Delta t\mu}{\left(\Delta x\right)^2}\left(\Delta u^n_{i+1}-2\Delta u^n_i+\Delta u^n_{i-1}\right)&=\mathrm{RHS_i^n} \\
    \displaystyle\Delta u_i^n+\theta\frac{\Delta t}{2\Delta x}\left(A_{i-1}^n\Delta u_{i-1}^n-A_{i+1}^n\Delta u_{i+1}^n\right)-\theta\frac{\Delta t\mu}{\left(\Delta x\right)^2}\left(\Delta u^n_{i+1}-2\Delta u^n_i+\Delta u^n_{i-1}\right)&=\mathrm{RHS_i^n} \\
    A_i\Delta u^n_{i-1}+B_i\Delta u^n_i+C_i\Delta u^n_{i+1}&=D_i
\end{align*}
\begin{equation*}
    \Downarrow
\end{equation*}
\begin{equation}
    \begin{array}{lcl}
        A_i &=& \displaystyle-\theta\frac{\mu\Delta t}{\left(\Delta x\right)^2}+\theta\frac{\Delta t}{2\Delta x}A_{i-1}^n\\
        B_i &=& \displaystyle1+2\theta\frac{\mu\Delta t}{\left(\Delta x\right)^2}\\
        C_i &=& \displaystyle-\alpha\frac{\mu\Delta t}{\left(\Delta x\right)^2}-\theta\frac{\Delta t}{2\Delta x}A_{i-1}^n\\
        D_i &=& RHS_i^n
    \end{array}
\end{equation}
and advancing the solution with:
\begin{equation}
    u_i^{n+1}=u_i^n+\Delta u_i^n
\end{equation}
In order to calculate $\Delta u^n_i$ it is needed to invert matrix as follows:
\begin{equation}
    \begin{pmatrix}
        B_1 & C_1 & 0 & \cdots & \cdots & \cdots & 0 \\
        A_2 & B_2 & C_2 & 0 & \cdots & \cdots & 0 \\
        0 & \ddots & \ddots & \ddots & 0 & \cdots & 0 \\
        0 & 0 & A_i & B_i & C_i & 0 & 0 \\
        0 & \cdots & 0 & \ddots & \ddots & \ddots & 0 \\
        0 & \cdots & \cdots & 0 & A_{N-2} & B_{N-2} & C_{N-2} \\
        0 & 0 & \cdots & \cdots & 0 & A_{N-1} &B_{N-1}
    \end{pmatrix}
    \begin{pmatrix}
        \Delta u_1^n\\
        \Delta u_2^n\\
        \cdots\\
        \cdots\\
        \cdots\\
        \Delta u_{N-2}^n\\
        \Delta u_{N-1}^n
    \end{pmatrix}
    =
    \begin{pmatrix}
        D_1-A_1\cdot u_0\\
        D_2\\
        \cdots\\
        \cdots\\
        \cdots\\
        D_{N-2}\\
        D_{N-1}-C_{N-1}\cdot u_N
    \end{pmatrix}
\end{equation}

The Beam and Warming method is extremely dispersive and therefore artiricial viscousity must be explicitly added. Beam and Warming used the following artificial viscosity term:
\begin{equation}
    \begin{matrix}
        \displaystyle-\frac{w}{8}\left(u^n_{i+2}-4u^n_{i+1}+6u^n_i-4u^n_{i-1}+u^n_{i-2}\right) && 0<w\le1
    \end{matrix}
\end{equation}
which can be added to the RHS with no change in accuracy

\end{document}